%\part{Malayalam Khuthubas}
\selectlanguage{malayalam}
\chapter{\textmalayalam{ഇബ്രാഹീം നബി} \textarabic{عليه السلام} \textmalayalam{യുടെ ഗുണഗണങ്ങൾ}}

\selectlanguage{arabic}

{السلام عليكم ورحمه الله وبركاته} \\

%
%إِنَّ الْحَمْدَ لِلَّهِ - نَحْمَدُهُ وَنَسْتَعِينُهُ وَنَسْتَغْفِرُهُ، وَنَعُوذُ بِاللَّهِ مِنْ شُرُورِ أَنْفُسِنَا، مَنْ يَهْدِهِ اللَّهُ فَلَا مُضِلَّ لَهُ، وَمَنْ يُضْلِلْ فَلَا هَادِيَ لَهُ، أَشْهَدُ أَنْ لَا إِلَهَ إِلَّا اللَّهُ، وَأَشْهَدُ أَنَّ مُحَمَّدًا عَبْدُهُ وَرَسُولُهُ ". ثُمَّ يَقْرَأُ ثَلَاثَ آيَاتٍ : { يَا أَيُّهَا الَّذِينَ آمَنُوا اتَّقُوا اللَّهَ حَقّ تقاته ولا تموتن إلا وأنتم مسلمون } ، { يا أيها الناس اتقوا ربكم الذي خلقكم من نفس واحدة وخلق منها زوجها وبث منهما رجالا كثيرا ونساء واتقوا الله الذي تساءلون به والأرحام إن الله كان عليكم رقيبا } ، { يا أيها الذين آمنوا اتقوا الله وقولوا قولا سديدا } { يصلح لكم أعمالكم ويغفر لكم ذنوبكم ومن يطع الله ورسوله فقد فاز فوزا عظيما } \\


 ان الحمدلله - نحمده ونستعينه ونستغفره، ونعوذ بالله من شرور انفسنا ومن سيئات اعمالنا، من يهده الله فلا مضل له ، ومن يضلل فلا هادي له، واشهد ان لا اله الا الله وحده لا شريك له، واشهد ان محمدا عبده ورسوله ، \\ 
 
 \quranayah[3][102] 
 ~ \\
  \quranayah[4][1] 
  ~ \\
    \quranayah[33][70] 
     \quranayah[33][71] 
    ~ \\
 اما بعد  \\ 
 
 فان اصدق الحديث كتاب الله ، وخير الهدي هدي محمد صلى الله عليه وعلى اله وصحبه وسلم ، وشر الامور محدثاتها ، و كل محدثه بدعه ، و كل بدعه ظلاله ، و كل ظلاله في النار عباد الله. \\ 
 
 
 
 \selectlanguage{malayalam}
 ഖലീലുല്ലാഹി ഇബ്രാഹിം  \textarabic{عليه السلام} അല്ലാഹു തആലാ മുഹമ്മദ് നബിക്കും അദ്ദേഹത്തിന്റെ ഉമ്മത്തിനും ഏറ്റവും അടുത്ത ബന്ധം ഉള്ളവരായി,  ഏറ്റവും നല്ല മാതൃകയായി എടുത്തു കാണിച്ചു തന്നിട്ടുള്ള വ്യക്തിയാണ്. \\
 
 സൂറത്തുൽ മുംതഹനയിൽ അല്ലാഹു  പറഞ്ഞു 
 
\textarabic{\quranayah[60][4]}
  
 ഇബ്രാഹീം നബിയിലും  അദ്ദേഹത്തിൻറെ കൂടെ ഉള്ള അനുചരന്മാരിലും നിങ്ങൾക്ക് ഏറ്റവും ഉത്തമമായ, ഉദാത്തമായ മാതൃകയുണ്ട്. \\
 
 സൂറത് ആലു ഇമ്രാനിൽ അറുപത്തിയെട്ടാമത് ആയത്തിൽ അല്ലാഹു പറഞ്ഞു, 
 
 \textarabic{\quranayah[3][68]}
 
 ഇബ്രാഹീം നബിയുമായി ഏറ്റവും അടുത്ത ബന്ധമുള്ളവർ ജൂതന്മാറോ ക്രിസ്ത്യാനികലോ മുശ്രിക്കുകളോ അല്ല,  ഇബ്രാഹീം നബിയുമായി  ആദർശപരമായ ബന്ധമുള്ളത് അദ്ദേഹത്തെ പിന്തുടർന്ന അദ്ദേഹത്തിൻറെ അനുചരന്മാരും ഈ നബി, അഥവാ മുഹമ്മദ് നബിയും  അദ്ദേഹത്തിൽ ഈമാൻ ഉൾക്കൊണ്ടിട്ടുള്ള വിശ്വാസികളും ആകുന്നു. അല്ലാഹു ആണ് മുഅമിനീങ്ങളുടെ വലിയ്യു, രക്ഷാധികാരി. \\
 
 
  ഇബ്രാഹിം നബി അലൈഹിസ്സലാം നമുക്ക് മാതൃകയാണ്. അദ്ദേഹത്തെ മാതൃകയാക്കി കൊണ്ടാണ് ധാരാളക്കണക്കിനു ഇബാദത്തുകൾ നമുക്ക് നിശ്ചയിക്കപ്പെട്ടിട്ടുള്ളത്. എന്നല്ല നിത്യേന നമ്മുടെ നമസ്കാരങ്ങളിലും റസൂലുല്ലാഹി -SS- ക്കു വേണ്ടി സ്വലാത് ചൊല്ലുന്ന സന്ദർഭങ്ങളിൽ, ഇബ്രാഹീമിയ സ്വലാത്ത്, ആ സ്വലാത്തിന്റെ പേര് തന്നെ ഇബ്രാഹീമിയ സ്വലാത്ത് എന്നാണു. അതു ഉച്ചരിച്ചു കൊണ്ടാണ് നമ്മുടെ നമസ്കാരം പോലും നാം  നിർവഹിക്കുന്നത്.  അത്രയും അടുത്ത ബന്ധമുള്ളവരാണ് ഇബ്രാഹിം നബിയുമായി നമ്മൾ.  \\ 
  
%  അദ്ദേഹത്തിൻറെ ഔസാഫുകൾ 
  ഒരു നബിയോട് നമുക്കുള്ള ബന്ധം,  ഒരു വ്യക്തിയോടു നമുക്കുള്ള ബന്ധം, അദ്ദേഹത്തോടുള്ള അടുപ്പം, അദ്ദേഹത്തോടുള്ള മാതൃക, അദ്ദേഹത്തിൻറെ ഗുണഗണങ്ങളും അത് നമ്മൾ അതുപോലെ ജീവിതത്തിൽ പകർത്തുക എന്നതിലാണ്. \\
  
  
  വാദിക്കുന്നതിലല്ല, സ്നേഹം ജല്പിക്കുന്നതിലല്ല, നടിക്കുന്നതിലല്ല, നേരെ മറിച്ചു സ്നേഹവും ബന്ധവും എന്ന് പറയുന്നത് അവരെ അതുപോലെതന്നെ മാതൃകയാക്കുന്നതിലാണ്. \\
  
   ഇബ്‌റാഹീം നബിയുടെ ഔസാഫുകൾ,  ഗുണഗണങ്ങൾ ധാരാളക്കണക്കിനു അല്ലാഹു നമുക്ക് മുജുമലായും മുഫസ്സലായും, സംക്ഷിപ്തമായും  വിശദമായും, നമുക്ക് വിവരിച്ചുതരുന്നുണ്ടു. \\
  
  റസുലുള്ളാഹി -SS- ധാരാളക്കണക്കിനു ഹദീസുകളിൽ ഇബ്‌റാഹീം നബിയുടെ  ഗുണങ്ങളും, അദ്ദേഹത്തിൻറെ മാതൃകയും ,അദ്ദേഹത്തിൻറെ കഥകളും നമുക്ക് വിവരിച്ചു തരുന്നുണ്ട്. ഇത് നമുക്ക് മാതൃകയായി ജീവിക്കുന്നതിനു വേണ്ടിയാണ്. \\ 
  
   സൂറത്തു നഹ്‌ലിന്റെ അവസാനഭാഗത്ത് നാല് കാര്യങ്ങൾ പ്രത്യേകം അല്ലാഹു ഉണർത്തുന്നത് കാണാം. \\
  
  
  \textarabic{\quranayah[16][120]}
    \textarabic{\quranayah[16][121]}
 
 നാലു ഇനം ഗുണങ്ങൾ അദ്ദേഹത്തിന് ചേർത്തിരിക്കുന്നു എന്നാണ് ഇബനുൽ ഖയ്യിം -R-   പറഞ്ഞിരിക്കുന്നത്.  
 
 1. ഇബ്‌റാഹീം നബി ഒരു ഉമ്മത്തു ആയിരുന്നു. \\ 
 ഇബ്നു ജരീര് അതബരീ, ഇബ്നു കസീർ, ബഗവീ ഇവരെല്ലാവരും സലഫകളെ, അബ്ദുല്ലാഹിബ്നു മസ്ഊദ്,  അബ്ദുല്ലാഹിബ്നു ഉമർ, മുജാഹിദ്, തുടങ്ങിയുള്ളവരെ ഉദ്ധരിച്ചു കൊണ്ട് വിവരിക്കുന്നതിന്റെ ചുരുക്കം ഇതാണ്. \\
  ഉമ്മത് എന്ന് പറഞ്ഞാൽ
   
 \textarabic{الإمام الذي يقتدى به} 
  
 ദീനിൽ മാതൃകയാക്കാൻ മുന്നിൽ നടന്നു പോകുന്നവരാണ്.  അവരെ നമുക്ക് റോൾ മോഡൽ ആക്കി അവരെ പ്പോലെ ആകാം. \\
 
 ഇമാം എന്ന വാക്കിനെക്കാൾ അർത്ഥപൂർണമായ വാക്കാണ് ഉമ്മത് എന്നത്. ഇബ്‌നുൽ ഖയ്യിം മിഫ്താഹ് സആഥയിൽ ആ രണ്ടു വാക്കുകൾ തമ്മിലുള്ള വ്യത്യാസം സുദീർഘമായി വിവരിക്കുന്നത് കാണാം.   ഇമാം എന്ന വാക്കിനെക്കാൾ  അക്ഷരങ്ങൾ കൊണ്ട് അധികം ഉള്ളതാണ് ഉമ്മത് എന്ന വാക്ക്.  എന്നത് പോലെ തന്നെ അതിന്റെ ആശയവും അത്രമാത്രം വ്യാപ്തി ഉള്ളതാണ്. \\
 
 ഒരാൾ ഉദ്ദേശപൂർവ്വമോ അല്ലാതെയോ മുന്നിൽ നടന്നു പോകുന്ന ഒരാളെ മാതൃക ആക്കിയാൽ അയാളെ നമ്മൾ ഇമാം ആക്കി എന്ന് പറയാം. ഉദ്ദേശപൂർവമാവണം എന്നില്ല. അത്പോലെ തന്നെ വഴിക്ക്, വഴി എന്നതിന് അറബി ഭാഷയിൽ  
 
  \textarabic{ \quranayah[15][79]}
        
  "അല്ലാഹുവിന്റെ ശിക്ഷകൾ സംഭവിച്ചതിന്റെ അടയാളങ്ങൾ നിലനിൽക്കുന്നത് അവർ കടന്നു പോകുന്ന വഴിയിൽ തന്നെയാണ്" എന്നതിന് ഇമാമു എന്ന് പ്രയോഗിച്ചത് കാണാം. \\ 
  
  വഴി എന്നത് ആളുകൾ പോയി പോയി ഉദ്ദേശപൂർവ്വമല്ലാതെ ഒരാൾ മറ്റൊരാളെ പിന്തുടർന്നു പോകുന്ന സ്ഥലമാണ്. \\ 
  ഉമ്മത്തു എന്നത് അങ്ങനെ അല്ല. ഉദ്ദേശപൂര്വ്വം മാതൃക ആക്കാൻ യോഗ്യമായവർക്കേ ഉമ്മത് എന്നത് ഉപയോഗിക്കൂ. \\
  
   മാത്രമല്ല, ഉമ്മത് എന്ന വക്കിൽ രണ്ടു മീമുകൾ ളമ്മത്തു കൊണ്ടാണ് ഉള്ളത്. അത് അതിന്റെ ശക്തിയെ കാണിക്കുന്നതാണ്, അതിന്റെ അര്ഥത്തിന്റെ ആഴത്തിനെ കാണിക്കുന്നതാണ്. \\ 
   
   അത് പോലെ  തന്നെ ളമ്മത്തു കൊണ്ടാണ് ഉമ്മത് എന്ന വാക്കു തുടങ്ങിയിട്ടുള്ളതും. ഇമാം എന്നത് കസറത്‌ കൊണ്ടാണ് തുടങ്ങുന്നത്. ളമ്മത്തുനു കസറത്തിനെക്കാൾ ബലം കൂടുതലാണ് ഭാഷയിൽ. \\
   
   അത് പോലെ തന്നെ, അവസാനത്തെ \textbf{ത}, \textbf{ത} കൊണ്ട് അവസാനിക്കുന്നത് \textarabic{افراد} നെ അറിയിക്കുന്നതാണ്. വ്യക്തിപരമായി ഒറ്റക്കാണ് അദ്ദേഹം പൊരുതിയത്. ഒറ്റ എന്ന് കാണിക്കുന്നതാണ് അവസാനത്തെ ത. ഒറ്റക്കുള്ള ഒരു റൂമിനു \textarabic{عرفة}- എന്ന് പറയുന്നത് പോലെ, ഒറ്റക്കുള്ള ഒരു വൃക്ഷത്തിന് \textarabic{شجرة}- എന്ന് പറയുന്നത് പോലെ. \\
     ഇങ്ങനെ ഇബ്നുൽ ഖയ്യിം ഭാഷാപരമായി ഉമ്മ എന്ന വാക്കിന് ഉള്ള അർത്ഥതിൻറെ ആഴം വിശദീകരിക്കുന്നത് കാണാം.\\
   
   
    ഇബ്‌റാഹീം നബി a- ഒരു ഉമ്മത് ആയിരുന്നു. ഖിയാമത്തു നാൾ വരെയുള്ള ജനങ്ങൾക്ക് മുഴുവനും  മാതൃകായോഗ്യനായ ഒരു സമുദായത്തിന്റെ സ്ഥാനത്തുള്ള വ്യക്തിയാണ്. ഒരു കൂട്ടത്തിന്റെ  സ്ഥാനത്തുള്ള വ്യക്തിയാണ്. അതാണ് ഉമ്മത്. \\
      
     
 മാത്രമല്ല അദ്ദേഹം ഏകനായിരുന്നു.  തന്റെ ആദർശത്തിൽ ഒറ്റക്ക് അടരാടിയ, പോരാടിയ ഒരു പോരാളിയായിരുന്നു അദ്ദേഹം. ഷെയ്ഖുൽ ഇസ്‌ലാം പറഞ്ഞത് പോലെ, \\
 \textarabic{كان مؤمنا وحده} 

കുഫ്ഫാറുകളായ ജനങ്ങൾക്കിടയിൽ മുഅമിനായി ഒറ്റക്ക് പോരാടിയ വ്യക്തിയാണ് അദ്ദേഹം. സ്വഹീഹുൽ ബുഖാരിയിൽ ഇങ്ങനെ കാണാം, ഭാര്യയായ സാറയോട് ഇബ്രാഹീം a- പറഞ്ഞു  

\textarabic{يَا سَارَةُ، لَيْسَ عَلَى وَجْهِ الْأَرْضِ مُؤْمِنٌ غَيْرِي وَغَيْرَكِ}

ഇന്നീ ഭൂലോകത് മുഉമിനായി ഞാനും നീയും അല്ലാതെ മറ്റാരും തന്നെ ഇല്ല. ഒറ്റക്ക് ഈമാനിൽ ഉറച്ചു നിന്നിരുന്ന വ്യക്തിയാണ് അദ്ദേഹം. \\

ഉമ്മത്താണ് അദ്ദേഹം. ഉമ്മത് എന്ന വാക്കിന് \textarabic{ابن مسعود} തഫ്സീർ നൽകിയത് \textarabic{معلم الخير}, \\
  ജനങ്ങൾക്ക് \textarabic{ الخير} പഠിപ്പിചു കൊടുക്കുന്ന വ്യക്തിയാണ് എന്നാണ്. \\
  
  

രണ്ടാമത്തെ വസ്ഫ് \textarabic{ أُمَّةً قَانِتًا},  ഖാനിത്തായിരുന്നു  \\
 \textarabic{والقانت : هو الخاشع المطيع} \\
 ഖാനിത്തു എന്ന വാക്കിന്റെ അർഥം അല്ലാഹുവിനെ അറിയേണ്ടത് പോലെ അറിഞ്ഞ്, അല്ലാഹുവിൻറെ വിധിവിലക്കുകൾ അറിഞ്ഞു, അല്ലാഹുവിൻറെ അസ്മാഉകളിലൂടെ, സ്വിഫാതുകളിലൂടെ ആണ്  അല്ലാഹുവിനെ മനസിലാക്കുക. അല്ലാഹുവിനെ അറിഞ്ഞു കൊണ്ട് ഭയപ്പെട്ടു ജീവിക്കുന്ന, അവന്റെ വിധിവിലക്കുകൾ പാലിക്കുന്ന \textarabic{الخاشع} ആയിരുന്നു \textarabic{المطيع} ആയിരുന്നു അദ്ദേഹം.\\
 
   അല്ലാഹുവിനു ധിക്കാരം കാണിക്കാത്ത, അല്ലാഹുവിൻറെ മുഴുവൻ വിധിവിലക്കുകാലും അനുസരിച്ചു ജീവിക്കുന്ന  \textarabic{المطيع} ആയിരുന്നു അദ്ദേഹം. \\
     ഷെയ്ഖുൽ ഇസ്‌ലാം പറയുന്നു \\ \textarabic{القنوت دوام الطاعة} \\ 
     ഖുനൂത് എന്നാൽ 
     എപ്പോയെങ്കിലും നിസ്കരിച്ചു,  എപ്പോയെങ്കിലും ഒരു അമൽ കുറെ അധികം ചെയ്തു. പിന്നെ കുറെ കാലത്തേക്കില്ല. ഇതല്ല, ഖുനൂത് ആവണമെങ്കിൽ -\textarabic{القنوت دوام الطاعة}- നിരന്തരമായി നിൽക്കുന്ന -\textarabic{الطاعة}- ആകണം. 
     
     \textarabic{الطاعة}-ൽ മുറിയാതെ,ജീവിതത്തിൽ ഒരിക്കൽ പോലും മുറിഞ്ഞു പോകാതെ നിരന്തരമായി അടിയുറച്ചു നിൽക്കുന്ന വ്യക്തി ആയിരുന്നു അദ്ദേഹം. 

\textarabic{وهو الذي يطيع الله دائماً}- എന്നെന്നും അല്ലാഹുവിനു -\textarabic{الطاعة}- ചെയ്യുന്ന വ്യക്തി ആയിരുന്നു അദ്ദേഹം. ഖാനിതായിരുന്നു. \\

  \textarabic{\quranayah[16][120]} 
  
  മൂന്നാമത്തെ വസ്ഫ് \textarabic{ قانتا لله حنيفا},  ഹനീഫായിരുന്നു അദ്ദേഹം. 
  
  ഹനീഫ് എന്ന വാക്കിന്റെ അർഥം 
  
  \textarabic{المنحرف قصدا عن الشرك إلى التوحيد}

  
  ശിർക്കിന്റെ എല്ലാ ഭാഗത്തുനിന്നും മാറുകയും, തൗഹീദിലേക്ക് മാത്രം ചായ്‌വ് കാണിക്കുകയും ചെയ്യുന്നവനാണ്. \\
  -\textarabic{قصدا}- ഉദ്ദേശപൂർവ്വം. \\
  കൃത്യമായി കാര്യങ്ങൾ പഠിച്ചറിഞ്ഞു തൗഹീദിന്മേൽ നില കൊള്ളുന്ന, ശിർക്കിന്റെ എല്ലാ തലങ്ങളിൽ നിന്നും, എല്ലാ ഇനങ്ങളിൽ നിന്നും, ശിർക്കിന്റെ എല്ലാ ഭാഗങ്ങളിൽ നിന്നും അങ്ങേയറ്റം വിട്ടുമാറി നിൽക്കുന്ന ഹനീഫായിരുന്നു അദ്ദേഹം. \\
  
  
 \textarabic{ولهذا قال : ( ولم يك من المشركين )} \\
  അത് കൊണ്ടാണ്‌  \textarabic{حنيفا} എന്ന് പറയുന്നതിൻറെ കൂടെ എപ്പോഴും  -\textarabic{ولم يك من المشركين}- മുശ്രിക്കുകളിൽ പെട്ടവനായിരുന്നീല്ല എന്ന് ചേർത്ത് പറയുന്നത്. 
  \\
  
 
 \textarabic{ أبي العبيدين} അദ്ദേഹം പറയുന്നു. \\
 \textarabic{أنه سأل عبد الله بن مسعود عن الأمة القانت} \\
 \textarabic{عبد الله بن مسعود} യോട് അദ്ദേഹം ചോദിച്ചു 
  \textarabic{ عن الأمة القانت}
  എന്താണ് -  \textarabic{  الأمة القانت}- എന്ന വാക്കിൻറെ അർത്ഥം. \\
 - \textarabic{عبد الله بن مسعود}- അദ്ദേഹം പറഞ്ഞു കൊടുത്തു \\
  \textarabic{فقال : الأمة : معلم الخير ، والقانت : المطيع لله ورسوله } \\
  ഉമ്മത് എന്ന് പറഞ്ഞാൽ -\textarabic{ معلم الخير}- 
  ജനങ്ങൾക്ക് -\textarabic{ الخير}- പഠിപ്പിക്കുന്നവൻ എന്നാണ്. 
  
  -\textarabic{القانت}- എന്നാൽ അല്ലാഹുവിനെയും റസൂലിനെയും അനുസരിക്കുന്നവൻ എന്നാണു. \\
 
 
 
 
 \textarabic{ ابن عمر} പറയുന്നു, \textarabic{الأمة الذي يعلم الناس دينهم .}
 
   \textarabic{الأمة} എന്ന വാക്കിൻറെ അർത്ഥം ജനങ്ങൾക്ക് -\textarabic{ دين}- പഠിപ്പിക്കുന്നവൻ എന്നാണ്. \\
 
 
 
 \textarabic{الشعبي} പറയുന്നു, \\
 \textarabic{فروة بن نوفل الأشجعي}  ഒരിക്കൽ \textarabic{عبد الله بن مسعود} പറയുന്നത് കേട്ടൂ. \\ 
 \textarabic{ قال ابن مسعود : إن معاذا كان أمة قانتا لله حنيفا}
 
 \textarabic{الأشجعي} പറയുന്നു \\
 \textarabic{ فقلت في نفسي} "എന്റെ മനസ്സിൽ ഞാൻ പറഞ്ഞു \\ -\textarabic{غلط أبو عبد الرحمن}- അബു അബ്‌ദുർറഹ്‌മാൻ അബദ്ധം പറഞ്ഞല്ലോ, തെറ്റി പറഞ്ഞല്ലോ. \\-\textarabic{إنما قال الله : ( إن إبراهيم كان أمة ) }- \\
 അല്ലാഹു പറഞ്ഞിട്ടുള്ളത് -\textarabic{إن إبراهيم كان أمة}-  എന്നല്ലേ. \\
 ഇബ്രാഹിം എന്ന് പറയേണ്ട സ്ഥാനത് മുആദ് എന്ന് പറഞ്ഞത് തെറ്റിപ്പോയാതാണോ.  \\
 \textarabic{فقال عبد الله بن مسعود}- \\
 പറഞ്ഞു 
 "മറന്നതല്ല, തെറ്റിയതല്ല, ബോധപൂർവ്വം പറഞ്ഞിട്ടുള്ളതാണ്" \\
 -\textarabic{أتدري ما الأمة وما القانت ؟}- \\
 -\textarabic{الأمة}- എന്ന വാക്കിൻറെ അർത്ഥം നിനക്കറിയുമോ? \\
 -\textarabic{القانت}- എന്ന വാക്കിൻറെ അർത്ഥം നിനക്കറിയുമോ? \\
 
 -\textarabic{قلت : الله [ ورسوله ] أعلم}- \\
 അദ്ദേഹം പറഞ്ഞു "എനിക്കറിയില്ല, അല്ലാഹുവാണ് ഏറ്റവും അറിയുന്നവൻ" \\
 
 -\textarabic{عبد الله بن مسعود}- പറഞ്ഞു \\
 \textarabic{ قال : الأمة الذي يعلم [ الناس ] الخير . والقانت : المطيع لله ورسوله} \\
   ഉമ്മത് എന്ന് പറഞ്ഞാൽ -\textarabic{ معلم الخير}- 
 ജനങ്ങൾക്ക് -\textarabic{ الخير}- പഠിപ്പിക്കുന്നവൻ എന്നാണ്. \\
 -\textarabic{القانت}- എന്നാൽ അല്ലാഹുവിനെയും റസൂലിനെയും അനുസരിക്കുന്നവൻ എന്നാണു. \\
 
 
 \textarabic{وكذلك كان معاذ معلم الخير ، وكان مطيعا لله ورسوله}
 
 അപ്രകാരമായിരുന്നു മുആദ് ബ്നു ജബൽ -r- . 
 ജനങ്ങൾക്ക് -\textarabic{الخير}- പഠിപ്പിക്കുന്ന, 
 സ്വഹാബത്തിന്റെ കൂട്ടത്തിൽ ഹലാൽ ഹറാമുകളെ കുറിച്ചു ഏറ്റവും നന്നായി അറിയുന്ന പണ്ഡിതനായിരുന്നു മുആദ് ബ്നു ജബൽ -r- . 
 അല്ലാഹുവിനെയും റസൂലിനെയും അനുസരിക്കുന്ന വ്യക്തി ആയിരുന്നു അദ്ദേഹം. \\
 
 അപ്രകാരമാണ് ഇബ്നു മസൂദ് -r- മുആദ് ബ്നു ജബൽ -r- .  ക്കുറിച്ചു വിശേഷിപ്പിച്ചിട്ടുള്ളത് \\
 
  \textarabic{ إن معاذا كان أمة قانتا لله حنيفا كما كان إبراهيم أمة}
 
 ജനങ്ങൾക്ക് നന്മ പഠിപ്പിക്കുന്ന അല്ലാഹുവിനെയും റസൂലിനെയും നിരന്തരമായി അനുസരണം പുലർത്തുന്ന വ്യക്തിയായിരുന്നു അദ്ദേഹം. \\
 
 \begin{center}


 \textarabic{اقول ما تسمعون واستغفر الله لي ولكم فاستغفروه انه هو الغفور الرحيم.} \\ 
   \end{center}
 	\hrule 
\pagebreak

 \textarabic{	الحمد لله الذي ارسل رسوله بالهدى ودين الحق ليظهره على الدين كله وكفى بالله شهيدا . واشهد ان لا اله الا الله وحده لا شريك له اقرارا بتروحين  .واشهد ان محمدا عبده ورسوله صلى الله عليه وعلى اله وسلم اما بعد الله 	
 	اتقوا الله يا عباد الله. }
 
 
 
 
 ഇബ്രാഹിം നബിയുടെ നാലാമത്തെ ഗുണമായി, വസ്ഫായി സൂറത്തു നഹ്‌ലിന്റെ അവസാനഭാഗത്ത് അല്ലാഹു പറയുന്നത് \\
 
 
 \textarabic{شاكرا لأنعمه}
 
 അല്ലാഹുവിൻറെ അനുഗ്രഹങ്ങൾക്ക് നന്ദി കാണിക്കുന്നവൻ ആയിരുന്നു അദ്ദേഹം. \\
 
  ഇബ്നുകസീർ പറയുന്നു \\
  \textarabic{أي : قائما بشكر نعم الله عليه ، كما قال}
 
 അല്ലാഹു അദ്ദേഹത്തിന് ചെയ്തു കൊടുത്തിട്ടുള്ള എല്ലാ അനുഗ്രഹങ്ങൾക്കും നന്ദി കാണിക്കുന്നവൻ ആയിരുന്നു അദ്ദേഹം.  അല്ലാഹു പറഞ്ഞതുപോലെ \\
   \textarabic{\quranayah[53][37]} 
 സൂറ നജ്ദിൽ 37 ആമത്തെ ആയത്തിൽ അല്ലാഹു ഇബ്രാഹിം നബിയുടെ വസ്തു പറഞ്ഞു 
 
 ഇബ്രാഹിം നബി, അദ്ദേഹം കരാർ പൂർത്തീകരിക്കുന്ന, തെന്റെ മേലുള്ളതു  പൂർത്തീകരിക്കുന്ന വനായിരുന്നു.  അല്ലാഹു ഒരു അനുഗ്രഹം  ചെയ്തു  കൊടുത്താൽ അനുഗ്രഹത്തിനു നന്ദി കാണിക്കുന്നവൻ,  അല്ലാഹു പരീക്ഷിച്ചാൽ ആ പരീക്ഷണത്തിന് ക്ഷമിക്കുന്നവൻ,  അതായിരുന്നു ഇബ്രാഹീം നബി. \\
 
  \textarabic{شاكرا لأنعمه} \\
  
  \textarabic{قام بجميع ما أمره الله تعالى به} \\
 അല്ലാഹു അദ്ദേഹത്തോട് കൽപ്പിച്ച മുഴുവൻ കൽപനകളും ജീവിതത്തിൽ പാലിച്ചവനാണ് ഇബ്രാഹീം നബി. \\
 
 വെറുതെ ഒരാൾക്കും ഒരു ദറജയും ദുനിയാവിലും പരലോകത്തും ലഭിക്കില്ല.  ഇബ്രാഹിം നബിക്കു ഈ ദറജകൽ മുഴുവൻ ലഭിച്ചത് 
 അദ്ദേഹത്തിൻറെ ഗുണഗണങ്ങൾ ഇപ്രകാരമായിരുന്നു എന്നതിനാലാണ് \\
 
 സൂറ ബഖറയിൽ അല്ലാഹു പറയുന്നു \\
    \textarabic{\quranayah[2][124]} 
    
    
    ഇബ്രാഹിം നബിയെ അല്ലാഹു ചില കലിമത്തുകൾ കൊണ്ട്, കൽപ്പനകൾ കൊണ്ട് പരീക്ഷിച്ചു.  ആ പരീക്ഷണങ്ങൾ നന്ദിപൂർവ്വം പൂർത്തീകരിക്കുകയാണ് 
    അദ്ദേഹം ചെയ്തിട്ടുള്ളത്. \\
    
    
അങ്ങനെ ത്യാഗം ചെയ്തു,  അല്ലാഹുവിൻറെ കൽപ്പനകൾ കൃത്യമായി ശിരസാവഹിച്ചു,  പരിപൂർണ്ണമായി അല്ലാഹുവിന് മാത്രം കീഴ്പ്പെട്ട്, അല്ലാഹുവിൻറെ വഹ്‌ദാനിയ്യത്തു അംഗീകരിച്ചു,  അല്ലാഹുവിന് ഹനീഫായി, ശിർക്ക് വെക്കാതെ, തൗഹീദിൽ അടിയുറച്ചു നിന്നു. 

ഇങ്ങനെയാണ് അല്ലാഹുവിന് 
അദ്ദേഹത്തിന് ഇമാമത് നൽകുന്നത്,
അദ്ദേഹത്തെ ഒരു ഉമ്മതാക്കുന്നത്,
അദ്ദേഹത്തിന് പദവികൾ നൽകുന്നത്,
അദ്ദേഹത്തിന് ഉയർച്ച നൽകുന്നത്, 

ദുനിയാവിലും പരലോകത്തും 
അല്ലാഹു നൽകിയിട്ടുള്ള മുഴുവൻ അനുഗ്രഹങ്ങളും അദ്ദേഹത്തിൻറെ ഔസാഫുകൽ  കൊണ്ടാണ്. \\


ഇബ്നുൽ ഖയ്യിം ഈ നാല് ഗുണങ്ങളും വിവരിച്ചുകൊണ്ട് പറയുന്നു. 

\textarabic{أنه سبحانه مدح خليله بأربع صفات كلها ترجع إلى العلم والعمل}


 നാല് ഗുണങ്ങൾ കൊണ്ട് അല്ലാഹു അവന്റെ ഖലീലായ ഇബ്രാഹിം നബിയെ സ്തുതിച്ചു കൊണ്ട് പറഞ്ഞു 
ഈ നാല് ഗുണങ്ങളും അടങ്ങുന്നത് രണ്ട് കാര്യങ്ങളിലേക്കാണ്. \\
\textarabic{علم} ലേക്കും \textarabic{عمل} ലേക്കും ആണ്. \\ അറിവിലേക്കും പ്രവർത്തിയിലേക്കും ആണ്. \\

\textarabic{بموجبه وتعليمه ونشره، فعاد الكمال كله إلى العلم والعمل بموجبه ودعوة الخلق إليه}
 
 

ഇൽമ് പഠിക്കുകയും, അതനുസരിച്ച് അമൽ ചെയ്യുകയും, ജനങ്ങളെ അത് പഠിപ്പിച്ചുകൊണ്ട് അതിലേക്ക് ക്ഷണിക്കുകയും, അല്ലാഹുവിൻറെ ദീൻ ദഅവത്തു ചെയ്യുകയും, എന്നതിലാണ്  എല്ലാ നന്മകളും നിലകൊള്ളുന്നത്. \\


അതായിരുന്നു ഇബ്രാഹിം നബിയുടെ ഗുണഗണങ്ങളിൽ ഏറ്റവും പ്രധാനപ്പെട്ട നാല് വസ്‌ഫുകൾ. 
അല്ലാഹു നമുക്ക് മാതൃകയാക്കി എടുത്തു കാണിച്ചു തന്നിട്ടുള്ളത് അത് കൊണ്ടാണ്. \\

ഇബ്രാഹിം നബിയെ അല്ലാഹു പ്രത്യേകം തിരഞ്ഞെടുത്തു,  

നബിയാക്കി, റസൂൽ ആക്കി, 
അദ്ദേഹത്തിന് മലക്കൂത്തു കാണിച്ചുകൊടുത്തു , അദ്ദേഹത്തെ ഖലീൽ ആയി സ്വീകരിച്ചു. 
ഇങ്ങനെ അല്ലാഹു ഈ ദുനിയാവിൽ ഒരു മനുഷ്യനു ലഭിക്കാവുന്ന ഏറ്റവും ഉയർന്ന പദവികളും നൽകി കൊണ്ട് അദ്ദേഹത്തെ തിരഞ്ഞെടുത്തു. 

    \textarabic{\quranayah[21][51]} 

എന്ന് സൂറത്തുൽ അമ്പിയാഇൽ പറഞ്ഞത് പോലെ, ഇബ്‌റാഹീം നബിക്കു അദ്ധേഹത്തിന്റെ റുശ്ദ് എത്തിയ സമയത്തു അല്ലാഹു നുബുവ്വത് നൽകി കൊണ്ട് തിരഞ്ഞെടുത്തു.

\textarabic{ثم قال : ( وهداه إلى صراط مستقيم )}
സിറാത്തുൽ മുസ്തഖീമിലേക്ക് അദ്ദേഹത്തിന് നാം വഴി കാണിച്ചു കൊടുത്തു. \\

ഏതാണ് സിറാത്തുൽ മുസ്തഖീം. 
ഇബ്നു കസീർ പറയുന്നു


\textarabic{وهو عبادة الله وحده لا شريك له على شرع مرضي }

അല്ലാഹുവിന് മാത്രം ഇബാദത്ത് ചെയ്യലാണ്, അല്ലാഹുവിനുള്ള ഇബാദത്തിൽ ആരെയും പങ്കുചേർക്കാതിരിക്കുകയും ആണ്,  അല്ലാഹു തൃപ്തിപ്പെടുന്ന രൂപത്തിൽ, അല്ലാഹു കാണിച്ചു തന്നിട്ടുള്ള രൂപത്തിൽ മാത്രം വഹിയ്യ് ഇറക്കിയ രൂപത്തിൽ തന്നെ അല്ലാഹുവിനെ ഇബാദത്ത് ചെയ്യലാണ്. 

അതാണ് സിറാത്തുൽ മുസ്തഖീം. \\


അതുകൊണ്ടുതന്നെ അല്ലാഹു ഇബ്രാഹിം നബിക്ക് നൽകി 

    \textarabic{\quranayah[16][122]} 

ഈ ദുനിയാവിൽ അദ്ദേഹത്തിന് ഹസനത്‌കൾ നൽകി. 

\textarabic{أي : جمعنا له خير الدنيا من جميع ما يحتاج المؤمن إليه في إكمال حياته الطيبة}



ഈ ദുനിയാവിൽ അദ്ദേഹത്തിന് ഹസനത്‌കൾ നൽകി. 
ഈ ദുനിയാവിൽ ഒരു മുഉമിനീനു  ആവശ്യമുള്ള എല്ലാ നന്മകളും അല്ലാഹു അദ്ദേഹത്തിന് നൽകി. ദുനിയാവിൽ ഏറ്റവും ത്വയ്യിബമായ ജീവിതം ജീവിക്കുന്നതിനു ആവശ്യമായ എല്ലാ ഗുണങ്ങളും അനുഗ്രഹങ്ങളും അല്ലാഹു അദ്ദേഹത്തിന് നൽകി. \\

 മാത്രമല്ല 

\textarabic{وانه في الاخره لمن الصالحين.} 

ആഖിറത്തിൽ ആകട്ടെ, അദ്ദേഹം  സ്വാലിഹീങ്ങളിൽ പെട്ടവൻ ആണ്. 

ഈ ദുനിയാവിൽ അല്ലാഹു അദ്ദേഹത്തിന് ഹസനാത്തുകൾ നൽകി എന്നതിൻറെ തഫ്സീർ ആയി ഇമാം മുജാഹിദ് പറയുന്നു \\

\textarabic{ لسان صدق} \\


ദുനിയാവിൽ അദ്ദേഹത്തിനുശേഷം വന്നിട്ടുള്ള എല്ലാ ജനങ്ങളും അദ്ദേഹത്തെ സത്യസന്ധതയോടെ കൂടി സ്മരിക്കുന്നു. 
സത്യത്തിന്റെ ആളുകൾ സത്യത്തിന്റെ നാവുകൊണ്ട് ഇബ്രാഹിം നബിയെ കുറിച്ച് സത്യവാൻ എന്ന സാക്ഷ്യപ്പെടുത്തുന്നു. 

ദുനിയാവിൽ കിട്ടാവുന്നതിൽ വെച്ച് കിട്ടാവുന്ന ഏറ്റവും വലിയ നന്മ അതാണ് . \\

മുഖാതിലുബ്നു ഹയ്യാൻ പറഞ്ഞത് 
ഇബ്രാഹിം നബി സ്മരിക്കപ്പെടുന്നത് മുഹമ്മദ് നബി സ്മരിക്കപ്പെടുന്നത് കൂടെയാണ്, സ്വലാത്തിൻറെ സമയത്ത്. 

അത്  ദുനിയാവിൽ അദ്ദേഹത്തിൻറെ ദിക്ർ,  അദ്ദേഹത്തെക്കുറിച്ചുള്ള കീർത്തി , അദ്ദേഹത്തെക്കുറിച്ചുള്ള മദ്ഹ് അല്ലാഹു 
ദുനിയാവിൽ 
ഉയർത്തി കൊടുത്തതിന് തെളിവാണ്.  ദുനിയാവിൽ അദ്ദേഹത്തിന് ഹസനത്‌കൾ  നൽകി എന്ന് പറഞ്ഞത് 


അദ്ദേഹം പരലോകത്തിൽ സ്വാലിഹീങ്ങളിൽ പെട്ടവനാണ് \\

എന്നിട്ട് അല്ലാഹു പറയുന്നത് 

    \textarabic{\quranayah[16][123]} 
എന്നിട്ട് താങ്കൾക്ക് മുഹമ്മദ് നബിയെ,  വഹിയ് നൽകിയിട്ടുള്ളത് ഈ ഇബ്രാഹിം നബിയുടെ മില്ലത്ത് പിന്തുടരുന്നതിനാണ് . 

എങ്ങനെയുള്ള മില്ലത്ത് , ഹനീഫായിട്ടുള്ള ഇബ്രാഹിം നബിയുടെ. 

\textarabic{أي : ومن كماله وعظمته وصحة توحيده وطريقه} \\
ഇബ്രാഹിം നബിയുടെ വഴി ഏറ്റവും ശരിയായ വഴിയാണ്. 

\textarabic{ أنا أوحينا إليك - يا خاتم الرسل وسيد الأنبياء - : ( أن اتبع ملة إبراهيم حنيفا وما كان من المشركين )}

ഇബ്രാഹിം നബിയുടെ വഴി ഏറ്റവും ശരിയായ വഴിയാണ് , തൗഹീദിന്റെ വഴിയാണ്. അത് കൊണ്ടാണ് അവസാന നബിയായ മുഹമ്മദ് നബിയോട് ഇബ്രാഹിം നബിയുടെ വഴി പിന്തുടരാൻ പറഞ്ഞിട്ടുള്ളത്. \\

 സൂറത്ത് അൻആമിൽ  അല്ലാഹു പറഞ്ഞതുപോലെ, 

    \textarabic{\quranayah[6][161]} 

നബിയെ പറയുക, എൻറെ റബ്ബ് 
എന്നെ വഴി കാണിച്ചു തന്നിരിക്കുന്നു, 
സിറാത്തുൽ മുസ്തഖീമിലേക്ക്. 
\textarabic{دينا قيما ملة إبراهيم}
നേരായ, ശരിയായ ദീൻ,  ഇബ്രാഹിം നബിയുടെ ഹനീഫ് ആയ ഇബ്രാഹിം നബിയുടെ മില്ലത്താണ്. 
അദ്ദേഹം മുശ്രിക്കുകളിൽ പെട്ടവൻ ആയിരുന്നില്ല. \\


ഇബ്രാഹിം നബിയുടെ ഔസ്സാഫുകൾ, ഗുണഗണങ്ങൾ 
അല്ലാഹു അവൻറെ കിതാബിൽ വിവരിച്ചു തന്നിട്ടുള്ളത് പഠിക്കുകയും അതനുസരിച്ച് ജീവിക്കുകയും 
ഉൾക്കൊള്ളുകയും ജീവിതത്തിൽ പകർത്തുകയും മറ്റുള്ളവർക്ക് പഠിപ്പിച്ചു കൊടുക്കുകയും ചെയ്തു ഇബ്രാഹിം നബിയെ മാതൃകയാക്കി ജീവിച്ചു മരിക്കാൻ അല്ലാഹു നമുക്ക് തൗഫീഖ് നൽകട്ടെ \\


\textarabic{بارك الله لنا ولكم في القران الكريم ونفعنا بما فيه من الايات والذكر الحكيم و صلى الله و سلم و بارك على نبينا محمد و على اله و صحبه اجمعين والحمد لله رب العالمين و قوموا الى صلاتكم يرحمكم الله.}


 