\chapter{\textarabic{ } -- \textmalayalam{അടിസ്ഥാനകാര്യങ്ങൾ}}

\selectlanguage{malayalam}
\begin{itemize}	    \setlength{\itemsep}{5pt}
	\item അറബി ഭാഷയിലെ എല്ലാ പദങ്ങളും നാമങ്ങളും പുല്ലിംഗമോ സ്ത്രീലിംഗമോ ആണ് 
	\item അറബി ഭാഷയിലെ പദങ്ങൾ ഏകവചനം, ദ്വിവചനം, ബഹുവചനം, എന്നിങ്ങനെയാണ് 
	\item ഏകവചനം എന്നത് ഒരു വ്യക്തിയെയോ വസ്തുവിനെയോ സൂചിപ്പിക്കുന്നു 
	\item ദ്വിവചനം രണ്ടിനെയും, ബഹുവചനം രണ്ടിലധികം ഉള്ളതിനെയും സൂചിപ്പിക്കുന്നു 
	\item മുകളിൽ സൂചിപ്പിച്ചിട്ടുള്ള കാര്യങ്ങൾ മുൻനിർത്തി തയ്യാറാക്കിയിട്ടുള്ള താഴെ പട്ടിക നോക്കുക. ഇപ്രകാരമായിരിക്കും  ഈ പുസ്തകത്തിലുടനീളം  നാം ഉപയോഗിക്കുക
\end{itemize}

\begin{table}[]
	\begin{tabular}{|l|l|l||l|}
		\hline
		ബഹുവചനം                                                                              & ദ്വിവചനം                                                                       & ഏകവചനം                                                              &                                                                 \\ \hline \hline
		\begin{tabular}[c]{@{}l@{}}അവർ, അവരുടെ \\ (രണ്ടിലധികം  പുരുഷന്മാർ)\end{tabular}      & \begin{tabular}[c]{@{}l@{}}അവർ, അവരുടെ \\ (രണ്ട് പുരുഷന്മാർ)\end{tabular}      & \begin{tabular}[c]{@{}l@{}}അവൻ, അവൻ്റെ,\\  അത്, അതിൻറെ\end{tabular} & \begin{tabular}[c]{@{}l@{}}അപരർ,\\ പുല്ലിംഗം\end{tabular}       \\ \hline
		\begin{tabular}[c]{@{}l@{}}അവർ, അവരുടെ \\ (രണ്ടിലധികം സ്ത്രീകൾ)\end{tabular}         & \begin{tabular}[c]{@{}l@{}}അവർ, അവരുടെ \\ (രണ്ട് പുരുഷന്മാർ)\end{tabular}      & \begin{tabular}[c]{@{}l@{}}അവൾ,അവളുടെ,\\ അത്, അതിൻറെ\end{tabular}   & \begin{tabular}[c]{@{}l@{}}അപരർ,\\ സ്ത്രീലിംഗം\end{tabular}     \\ \hline
		\begin{tabular}[c]{@{}l@{}}നിങ്ങൾ, നിങ്ങളുടെ\\ (രണ്ടിലധികം  പുരുഷന്മാർ)\end{tabular} & \begin{tabular}[c]{@{}l@{}}നിങ്ങൾ, നിങ്ങളുടെ\\ (രണ്ട് പുരുഷന്മാർ)\end{tabular} & \begin{tabular}[c]{@{}l@{}}നീ, നിൻ്റെ\\ (പുരുഷൻ)\end{tabular}       & \begin{tabular}[c]{@{}l@{}}സംബോധിധർ,\\ പുല്ലിംഗം\end{tabular}   \\ \hline
		\begin{tabular}[c]{@{}l@{}}നിങ്ങൾ, നിങ്ങളുടെ\\ (രണ്ടിലധികം സ്ത്രീകൾ)\end{tabular}    & \begin{tabular}[c]{@{}l@{}}നിങ്ങൾ, നിങ്ങളുടെ\\ (രണ്ട് സ്ത്രീകൾ)\end{tabular}   & \begin{tabular}[c]{@{}l@{}}നീ, നിൻ്റെ\\ (സ്ത്രീ)\end{tabular}       & \begin{tabular}[c]{@{}l@{}}സംബോധിധർ,\\ സ്ത്രീലിംഗം\end{tabular} \\ \hline
		\multicolumn{2}{|l|}{ഞങ്ങൾ, ഞങ്ങളുടെ}                                                                                                                                 & ഞാൻ, എൻ്റെ                                                          & സംസാരിക്കുന്നവർ                                                 \\ \hline
	\end{tabular}
\end{table}